
\subsection{研究意义}
%%%%%%%%
%\newpage
\vspace{-5pt}
模版蓝色字部分没有问题了,正文自己写的排版可能不是你想要的,这种情况下,\textbf{多阅读一下nsfc.sty,可以解决你绝大部分问题}。以下是一些简单的排版示例。
\subsubsection{字体}
中文\textbf{粗体};\textbf{bold} font;
中文\textit{斜体};\textit{italic} font;

全文改宋体,可以修改nsfc.sty的MS部分字体。

可选的就是\verb|\zhkai,\enkai,\zhsong,\ensong|。

\subsubsection{文献}
普通引用\cite{test};上标引用\citess{test};多篇文章\citess{test,test2,test3}。

有注音的英文:\cite{test}。

参考期刊\cite{test};
参考图书\cite{test2};
参考会议\cite{test5};
参考链接\cite{test4};
参考文件\cite{test6}。

对于中文参考文献,bib条目中需要有language = {zh},参见\cite{test2}。
\subsubsection{列表}
值得注意的是,不需要一定要用列表环境,用加粗、换行、缩进同样能达到效果。
因为咱们的初衷,还是LaTeX在排版文献和公式上有优势,发挥这一个优势就行了,其他部分不需要强行套用。文本本身还是最重要、需要大家投入精力的部分。

无序列表的例子:
\begin{itemize}[left= 50pt]
	\item[-] 第一条,第一条的内容可能很长长长长长长长长长长长长长长长长长长长长;
	\item[-] 第二条。
\end{itemize}

有序列表的例子:
\begin{enumerate}[left= 50pt]
	\item 第一条,第一条的内容可能很长长长长长长长长长长长长长长长长长长长长;
	\item 第二条。
\end{enumerate}

两个带圈文字的实现方法:
\textcircled{\raisebox{-0.8pt}{1}}
\textcircled{\textbf{\small 1}}

注意,由于列表的缩进,不同使用者可能偏向并不一样。本模版用的enumitem包,阅读他的文档进行个性化,其文档在:https://www.ctan.org/pkg/enumitem
\subsubsection{图}
图片的例子:
\begin{figure}[h]
\centering %图片居中
\includegraphics[width=5cm]{figures/wechatgroup}
\captionsetup{justification=centering} %图题居中
\caption{这是图题。}
\end{figure}
图题和表头若想取消加粗,去掉nsfc.sty中caption部分的\verb|\bfseries|即可。


\subsubsection{表}
在表格内的第一行设置\verb|\zhkai\ensong\selectfont|,来选择字体。

其中\verb|\zhkai\zhsong\enkai\ensong|可以根据需要选择。
\begin{table}[htbp]
	\zhkai\ensong\selectfont%设置表格字体
	\centering  % 显示位置为中间
	\caption{表格}  % 表格标题
	\label{table1}  % 用于索引表格的标签
	%字母的个数对应列数,|代表分割线
	% l代表左对齐,c代表居中,r代表右对齐
	\begin{tabular}{|c|c|c|c|}  
		\hline  % 表格的横线
		& & & \\[-6pt]  %可以避免文字偏上来调整文字与上边界的距离
		第一列&第二列&第三列&第四列 \\  % 表格中的内容,用&分开,\\表示下一行
		\hline
		& & & \\[-6pt]  %可以避免文字偏上 
		0.1&0.2&0.3&0.4 \\
		\hline
	\end{tabular}
\end{table}

\subsubsection{公式}

\begin{equation}
	E=mc^2
\end{equation}

\begin{REF}
\subsection*{参考文献}
\vspace{-50pt}
\bibliographystyle{gbt7714-nsfc}
\bibliography{ref}%参考文献
\end{REF}

\newpage%自己判断是否需要
