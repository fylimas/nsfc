\documentclass[UTF8, punct, oneside, fontset=none]{ctexbook}
\usepackage[windows]{nsfc}
%【注意】在sections/个性化设置.sty中,设置修改选项。

\pagestyle{empty} % 第二页以后页码空白
\usepackage[a4paper, left = 3.18cm, right = 3.18cm, top = 2.54cm, bottom = 2.54cm]{geometry}%页边距

\usepackage{times}

\usepackage{setspace}

\usepackage{comment}

\graphicspath{{figures/}}   % 设置图片所存放的目录
\begin{document}
%\thispagestyle{empty}    % 首页页码空白

{\centering\zhkai\enkai\fontsize{16pt}{21pt}\selectfont
\setlength{\baselineskip}{21pt}%最后设置,防止被fontsize中的覆盖
\hspace{-11pt}\textbf{报告正文(2026版)}\par
\vspace{5.46pt}}%0.2(行距倍数)*21(行距)*1.3(word)

%\begin{comment}
{\zhkai\fontsize{14pt}{22pt}\selectfont
 \setlength{\baselineskip}{22pt}%最后设置,防止被fontsize中的覆盖
\textbf{\color{NSFblue}(请勿删除或改动下述提纲标题及括号中的文字)}\par
\vspace{.3pt}}%0.5(行距倍数)*22(行距)*1.3(word)
%\end{comment}

\chapter{\textbf{主要学术成绩、创新点及其科学意义}(建议不超过 4000 字)

\hspace{0.3pt}按本年度《国家自然科学基金项目指南》中青年科学基金项目(B 类)的有关要求,\kg{-0.2em}着重阐述所取得的研究成果的创新性和科学价值等。}

\begin{MS}
	
\subsection{研究意义}
%%%%%%%%
%\newpage
\vspace{-5pt}
\begin{REF}
\subsection*{参考文献}
\vspace{-50pt}
\bibliographystyle{gbt7714-nsfc}
  \bibliography{ref}%参考文献
\end{REF}


\end{MS}

\chapter{\textbf{拟开展的研究工作}(建议不超过 4000 字)

\hspace{0.3pt}着重阐述拟开展的研究工作的科学意义和创新性,技术路线、研 究方案等的可行性。}

\begin{MS}
	
%\newpage
\subsection{与本项目密切相关的研究工作积累}
申请人在xx较为深入的学习和探索,为本项目提供了重要基础和关键支撑。

\textcircled{\small 1} {\bfseries 在xx方面}:

行间距变化一般是在“多行蓝色模版”部分前后。因为蓝色模版文字在section里写的,latex把蓝色部分当作一个整体,
蓝色模版文字不会跨到两页上,导致会挤前一页的行间距,导致前一页行距异常。
针对这种情况的解决方法:
\begin{itemize}
	\item[(1)] 把它newpage到后一页上,就行了。也可以考虑分段缓解,需要写的时候进行规划。
	\item[(2)] 使用如下nsfc.sty中的相关代码:
	\begin{lstlisting}[language=tex, basicstyle=\ttfamily\small, keywordstyle=\color{blue}, commentstyle=\color{gray}]
		%自动段落的行间距微调
		\usepackage{setspace}
		\setstretch{1.6} % 22 bp / 14 pt = 1.571
	\end{lstlisting}
\end{itemize}

Lorem ipsum dolor sit amet, consectetur adipiscing elit. Vestibulum ultrices lectus eget nunc gravida, eget aliquam purus condimentum. Nullam feugiat, ligula a vestibulum porta, orci sem finibus massa, eu vestibulum risus ex eu metus. 
Lorem ipsum dolor sit amet, consectetur adipiscing elit. Vestibulum ultrices lectus eget nunc gravida, eget aliquam purus condimentum. Nullam feugiat, ligula a vestibulum porta, orci sem finibus massa, eu vestibulum risus ex eu metus.

Lorem ipsum dolor sit amet, consectetur adipiscing elit. Vestibulum ultrices lectus eget nunc gravida, eget aliquam purus condimentum. Nullam feugiat, ligula a vestibulum porta, orci sem finibus massa, eu vestibulum risus ex eu metus. 
Lorem ipsum dolor sit amet, consectetur adipiscing elit. Vestibulum ultrices lectus eget nunc gravida, eget aliquam purus condimentum. Nullam feugiat, ligula a vestibulum porta, orci sem finibus massa, eu vestibulum risus ex eu metus.

Lorem ipsum dolor sit amet, consectetur adipiscing elit. Vestibulum ultrices lectus eget nunc gravida, eget aliquam purus condimentum. Nullam feugiat, ligula a vestibulum porta, orci sem finibus massa, eu vestibulum risus ex eu metus.
\end{MS}

\chapter{\textbf{其他需要说明的情况}}
\section{申请人同年申请不同类型的国家自然科学基金项目情况(列明同年申请的其他项目的项目类型、项目名称信息,并说明与本项目之间的区别与联系;已收到自然科学基金委不予受理或不予资助决定的,无需列出)。}

\begin{MS}
	\input{sections/三1同年申请}
\end{MS}

\section{具有高级专业技术职务(职称)的申请人是否存在同年申请或者参与申请国家自然科学基金项目的单位不一致的情况;\kg{-0.1em}如存在上述情况,\kg{-0.1em}列明所涉及人员的姓名,\kg{-0.1em}申请或参与申请的其他项目的项目类型、\kg{-0.1em}项目名称、\kg{-0.1em}单位名称、\kg{-0.1em}上述人员在该项目中是申请人还是参与者,并说明单位不一致原因。}

\begin{MS}
	\input{sections/三2同年不一致}
\end{MS}

\section{具有高级专业技术职务(职称)的申请人是否存在与正在承担的国家自然科学基金项目的单位不一致的情况;如存在上述情况,列明所涉及人员的姓名,正在承担项目的批准号、项目类型、项目名称、单位名称、起止年月,并说明单位不一致原因。}

\begin{MS}
	\input{sections/三3已有不一致}
\end{MS}

\section{同年以不同专业技术职务(职称)申请或参与申请科学基金项目的情况(应详细说明原因)。}

\begin{MS}
	\input{sections/三4同年不同职务}
\end{MS}

\section{其他。}

\begin{MS}
	\input{sections/三5其他}
\end{MS}

\end{document}
