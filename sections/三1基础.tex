

这里可能需要列出自己的相关文章。由于文章和依据部分的文献的格式并不一定一致,可以考虑选择下边两种方法之一:

方法一:拿到bbl文件。这个东西从哪里来的呢?从编译产生的\verb|.bbl|文件中拷贝过来放进来,\textcolor{blue}{作适当修改},就可以了。
例如,拿到bbl文件如下:
\begin{lstlisting}[language=tex, basicstyle=\ttfamily\tiny, keywordstyle=\color{blue}, commentstyle=\color{gray}]
\begin{thebibliography}{1}
	\providecommand{\bibauthor}[1]{#1}
	\providecommand{\bibeditor}[1]{#1}
	\providecommand{\bibtranslator}[1]{#1}
	\providecommand{\bibtitle}[1]{#1}
	\providecommand{\bibbooktitle}[1]{#1}
	\providecommand{\bibjournal}[1]{#1}
	\providecommand{\bibmark}[1]{#1}
	\providecommand{\bibcountry}[1]{#1}
	\providecommand{\bibpatentid}[1]{#1}
	\providecommand{\bibedition}[1]{#1}
	\providecommand{\biborganization}[1]{#1}
	\providecommand{\bibaddress}[1]{#1}
	\providecommand{\bibpublisher}[1]{#1}
	\providecommand{\bibinstitution}[1]{#1}
	\providecommand{\bibschool}[1]{#1}
	\providecommand{\bibvolume}[1]{#1}
	\providecommand{\bibnumber}[1]{#1}
	\providecommand{\bibversion}[1]{#1}
	\providecommand{\bibpages}[1]{#1}
	\providecommand{\bibmodifydate}[1]{#1}
	\providecommand{\bibcitedate}[1]{#1}
	\providecommand{\bibyear}[1]{#1}
	\providecommand{\bibdate}[1]{#1}
	\providecommand{\biburl}[1]{\newline\url{#1}}
	\bibitem{test}
	\bibauthor{\textbf{C\"aldognetto T\@, Tenti P}}\@. \bibtitle{Microgrids Operation Based
		on Master–Slave Cooperative Control}\bibmark{[J/OL]}\@. \bibjournal{IEEE
		Journal of Emerging and Selected Topics in Power Electronics}\@,
	\bibvolume{2}\bibnumber{(4)}\thinspace{}\textnormal{:
	}\bibpages{1081\thinspace{}\textnormal{--}\thinspace{}1088}\@,
	\bibyear{2014}\@. \url{doi: 10.1109/JESTPE.2014.2345052}\@.
\end{thebibliography}
\end{lstlisting}
上述bbl文件修改如下:
\begin{lstlisting}[language=tex, basicstyle=\ttfamily\tiny, keywordstyle=\color{blue}, commentstyle=\color{gray}]
\vspace{-50pt}%【1】根据自己需要设置
\begin{thebibliography}{1}
	\providecommand{\bibauthor}[1]{#1}
	\providecommand{\bibeditor}[1]{#1}
	\providecommand{\bibtranslator}[1]{#1}
	\providecommand{\bibtitle}[1]{#1}
	\providecommand{\bibbooktitle}[1]{#1}
	\providecommand{\bibjournal}[1]{#1}
	\providecommand{\bibmark}[1]{#1}
	\providecommand{\bibcountry}[1]{#1}
	\providecommand{\bibpatentid}[1]{#1}
	\providecommand{\bibedition}[1]{#1}
	\providecommand{\biborganization}[1]{#1}
	\providecommand{\bibaddress}[1]{#1}
	\providecommand{\bibpublisher}[1]{#1}
	\providecommand{\bibinstitution}[1]{#1}
	\providecommand{\bibschool}[1]{#1}
	\providecommand{\bibvolume}[1]{#1}
	\providecommand{\bibnumber}[1]{#1}
	\providecommand{\bibversion}[1]{#1}
	\providecommand{\bibpages}[1]{#1}
	\providecommand{\bibmodifydate}[1]{#1}
	\providecommand{\bibcitedate}[1]{#1}
	\providecommand{\bibyear}[1]{#1}
	\providecommand{\bibdate}[1]{#1}
	\providecommand{\biburl}[1]{\newline\url{#1}}
	%\bibitem{test}%【2】去掉。或者需要保证全文没有重复的test标签
	[1] %【3】手动添加标号
	\bibauthor{\textbf{C\"aldognetto T\@, Tenti P}}\@. \bibtitle{Microgrids Operation Based
		on Master–Slave Cooperative Control}\bibmark{[J/OL]}\@. \bibjournal{IEEE
		Journal of Emerging and Selected Topics in Power Electronics}\@,
	\bibvolume{2}\bibnumber{(4)}\thinspace{}\textnormal{:
	}\bibpages{1081\thinspace{}\textnormal{--}\thinspace{}1088}\@,
	\bibyear{2014}\@. \url{doi: 10.1109/JESTPE.2014.2345052}\@.
\end{thebibliography}
\end{lstlisting}
编译之后效果如下:
\begin{REF}
\vspace{-50pt}%【1】根据自己需要设置
\begin{thebibliography}{1}
	\providecommand{\bibauthor}[1]{#1}
	\providecommand{\bibeditor}[1]{#1}
	\providecommand{\bibtranslator}[1]{#1}
	\providecommand{\bibtitle}[1]{#1}
	\providecommand{\bibbooktitle}[1]{#1}
	\providecommand{\bibjournal}[1]{#1}
	\providecommand{\bibmark}[1]{#1}
	\providecommand{\bibcountry}[1]{#1}
	\providecommand{\bibpatentid}[1]{#1}
	\providecommand{\bibedition}[1]{#1}
	\providecommand{\biborganization}[1]{#1}
	\providecommand{\bibaddress}[1]{#1}
	\providecommand{\bibpublisher}[1]{#1}
	\providecommand{\bibinstitution}[1]{#1}
	\providecommand{\bibschool}[1]{#1}
	\providecommand{\bibvolume}[1]{#1}
	\providecommand{\bibnumber}[1]{#1}
	\providecommand{\bibversion}[1]{#1}
	\providecommand{\bibpages}[1]{#1}
	\providecommand{\bibmodifydate}[1]{#1}
	\providecommand{\bibcitedate}[1]{#1}
	\providecommand{\bibyear}[1]{#1}
	\providecommand{\bibdate}[1]{#1}
	\providecommand{\biburl}[1]{\newline\url{#1}}
	\bibitem{test}%【2】如果全文有重复的test标签,需修改成不重复的,避免和依据部分文献编号混淆
	\bibauthor{\textbf{C\"aldognetto T\@, Tenti P}}\@. \bibtitle{Microgrids Operation Based
		on Master–Slave Cooperative Control}\bibmark{[J/OL]}\@. \bibjournal{IEEE
		Journal of Emerging and Selected Topics in Power Electronics}\@,
	\bibvolume{2}\bibnumber{(4)}\thinspace{}\textnormal{:
	}\bibpages{1081\thinspace{}\textnormal{--}\thinspace{}1088}\@,
	\bibyear{2014}\@. \url{doi: 10.1109/JESTPE.2014.2345052}\@.
\end{thebibliography}
\end{REF}


方法二:你也能发现,实际上latex编译参考文献的方法,是把bib文件编译成具有排版格式的bbl文件。既然如此,还可以自行手动编参考文献,也不太难。例如,把排版好的文献拿过来,手动加粗、斜体需要的部分,如下:
\begin{REF}
\begin{enumerate}[label={[\arabic*]}]
	\item \textbf{C\"aldognetto T, Tenti P}. 
	Microgrids Operation Basedon Master–Slave Cooperative Control[J/OL],
	 \textit{IEEE Journal of Emerging and Selected Topics in Power Electronics}, 2(4): 1081 1088, 2014. 
	 \url{doi: 10.1109/JESTPE.2014.2345052}.
\end{enumerate}
\end{REF}

